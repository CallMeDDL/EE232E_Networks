\documentclass[11pt]{article}
	\usepackage[T1]{fontenc}
    % Nicer default font (+ math font) than Computer Modern for most use cases
    % \usepackage{mathpazo}

    % Basic figure setup, for now with no caption control since it's done
    % automatically by Pandoc (which extracts ![](path) syntax from Markdown).
    \usepackage{graphics}
    % We will generate all images so they have a width \maxwidth. This means
    % that they will get their normal width if they fit onto the page, but
    % are scaled down if they would overflow the margins.
    \makeatletter
    \def\maxwidth{\ifdim\Gin@nat@width>\linewidth\linewidth
    \else\Gin@nat@width\fi}
    \makeatother
    \let\Oldincludegraphics\includegraphics
    % Set max figure width to be 80% of text width, for now hardcoded.
    \renewcommand{\includegraphics}[1]{\Oldincludegraphics[width=.8\maxwidth]{#1}}
    % Ensure that by default, figures have no caption (until we provide a
    % proper Figure object with a Caption API and a way to capture that
    % in the conversion process - todo).
    \usepackage[center,bf]{caption}
    % \DeclareCaptionLabelFormat{nolabel}{}
    % \captionsetup{labelformat=nolabel}

    \usepackage{adjustbox} % Used to constrain images to a maximum size 
    \usepackage{xcolor} % Allow colors to be defined
    \usepackage{enumerate} % Needed for markdown enumerations to work
    \usepackage{geometry} % Used to adjust the document margins
    \usepackage{amsmath} % Equations
    \usepackage{amssymb} % Equations
    \usepackage{textcomp} % defines textquotesingle
    % Hack from http://tex.stackexchange.com/a/47451/13684:
    \AtBeginDocument{%
        \def\PYZsq{\textquotesingle}% Upright quotes in Pygmentized code
    }
    \usepackage{upquote} % Upright quotes for verbatim code
    \usepackage{eurosym} % defines \euro
    \usepackage[mathletters]{ucs} % Extended unicode (utf-8) support
    \usepackage[utf8x]{inputenc} % Allow utf-8 characters in the tex document
    \usepackage{fancyvrb} % verbatim replacement that allows latex
    \usepackage{grffile} % extends the file name processing of package graphics 
                         % to support a larger range 
    % The hyperref package gives us a pdf with properly built
    % internal navigation ('pdf bookmarks' for the table of contents,
    % internal cross-reference links, web links for URLs, etc.)
    \usepackage{hyperref}
    \usepackage{longtable} % longtable support required by pandoc >1.10
    \usepackage{booktabs}  % table support for pandoc > 1.12.2
    \usepackage[inline]{enumitem} % IRkernel/repr support (it uses the enumerate* environment)
    \usepackage[normalem]{ulem} % ulem is needed to support strikethroughs (\sout)
                                % normalem makes italics be italics, not underlines
   	\usepackage[]{authblk}
   	\usepackage{cite}
    \usepackage{graphicx}
    \usepackage{hyperref}
    \usepackage{amsmath}
    \usepackage{amsthm}
    \usepackage{amssymb}
    \usepackage{bm}
    \usepackage{bbm}
    \usepackage{algorithmicx}
    \usepackage{algorithm}
    \usepackage{algpseudocode}
    \usepackage{array}
    \usepackage{booktabs}
    \usepackage{multirow}
    \usepackage{makecell}
    \usepackage{color}
    \usepackage{tabularx,ragged2e,booktabs,caption}

   	\makeatletter
    \def\@maketitle{%
    \newpage
      \null
      \vskip 2em%
      \begin{center}%
      \let \footnote \thanks
        {\Large\bfseries \@title \par}%
        \vskip 1.5em%
        {\normalsize
          \lineskip .5em%
          \begin{tabular}[t]{c}%
            \@author
          \end{tabular}\par}%
        \vskip 1em%
        {\normalsize \@date}%
      \end{center}%
      \par
      \vskip 1.5em}
    \makeatother


\newtheorem{theorem}{Theorem}






\title{EE 232E Project 2\\Social Network Mining}
\author{Hengjie~Yang, Sheng~Chang, Wendi~Cui, and Tianyi~Liu
}


\date{\today}


\begin{document}
\maketitle

% \section{A brief tutorial on how to use this template}
% \Large\textcolor{red}{\bf{Please remove the tutorial section in the final manuscript\\ by commenting, i.e. $\%(something)$}}


% \subsection{Figures}
% Figure insertion is shown in Fig \ref{example_fig}.
% \begin{figure}[h]
% \centering
% \scalebox{0.7}{\includegraphics{Figures/spectrum_bar.pdf}}
% \caption{An example of figure insertion}
% \label{example_fig}
% \end{figure}

% \subsection{Equations}
% An example of equations is given as follows.
% \begin{theorem}
% Let $a$, $b$, $c$ denote the sides of a triangle, respectively. If $a\perp b$, the pythagoras theorem is given as follows.
% \begin{align}
% c^2 = a^2 + b^2
% \end{align}
% \end{theorem}

% \subsection{Tables}
% An example of tables is shown in Table \ref{example_table}.
% \renewcommand\arraystretch{1.1}
% \begin{table}[h]
% \center
% \caption{Standard CRC Codes versus Optimal CRC Codes for Convolutional Code $G=(561~753)$ with $n=504$ Bits}
% \scalebox{0.9}{
% \begin{tabular}{r|c|c|cccccc}
% \hline
% \multirow{2}{*}{Name} & \multirow{2}{*}{Gen. Poly.} & \multicolumn{7}{c}{Undetected Error Distance Spectrum} \\
% \cline{3-9}
%  & & $d$ & 16 & 18 & 20 & 22 & 24 & 26 \\\hline\hline
% Standard-8 & \multicolumn{1}{l}{0x19B} & & 983 & 4387 & 19909 & 105000 & 672724 & 3972970\\
% Optimal-8 & \multicolumn{1}{l}{0x19D} & & 0 & 979 & 22349 & 111304 & 686314 & 3830340\\\hline
% Standard-12 & \multicolumn{1}{l}{0x180F} & & 0 & 0 & 969 & 5815 & 42893 & 245211 \\
% Optimal-12 & \multicolumn{1}{l}{0x108B} & & 0 & 0 & 0 & 4793 & 45795 & 246729\\\hline
% Standard-16 & \multicolumn{1}{l}{0x11021} & & 0 & 0 & 484 & 0 & 1765 & 14752\\
% Optimal-16 & \multicolumn{1}{l}{0x1F8FD} & & 0 & 0 & 0 & 0 & 0 & 13240\\\hline
% \end{tabular}}
% \label{example_table}
% \end{table}


\section{Facebook network}

\subsection{Structural properties of the facebook network}

The facebook network is plotted in Fig. \ref{fig:facebook_network}.
\begin{figure}[h]
\centering
\scalebox{0.7}{\includegraphics{Figures/hj_01}}
\caption{Facebook network}
\label{fig:facebook_network}
\end{figure}

\textcolor{red}{Question 1: Is the facebook network connected? If not, find the giant connected component (GCC) of the network and report the size of the GCC.}

The facebook network is connected.\\

\textcolor{red}{Question 2: Find the diameter of the network. If the network is not connected, then find the diameter of the GCC.}

The diameter of the Facebook network is $8$.\\

\textcolor{red}{Question 3: Plot the degree distribution of the facebook network and report the average degree.}

The degree distribution of the Facebook network is shown in Fig. \ref{fig:deg_distribution}. The average degree is $43.69101$.\\
\begin{figure}[t]
\centering
\scalebox{0.7}{\includegraphics{Figures/hj_02}}
\caption{Degree distribution of the Facebook network}
\label{fig:deg_distribution}
\end{figure}

\textcolor{red}{Question 4: Plot the degree distribution of question 3 in a log-log scale. Try to fit a line to the plot and estimate the slope of the line.}

The degree distribution of Facebook network in log-log scale is shown in Fig. \ref{fig:deg_distribution_log}. In order to find a line which fits the data, we consider the data starting from $20$-th and ends $6$ before the end, thus, with linear regression analysis, the line we find is as follows.
\begin{align}
y=1.032-1.607x
\end{align}
where $y$ represents the $\log(frequency)$ and $x$ represents the $\log(degree)$. The estimated slope is $-1.607$.
\begin{figure}[t]
\centering
\scalebox{0.8}{\includegraphics{Figures/hj_03}}
\caption{Degree distribution of the Facebook network in log-log scale}
\label{fig:deg_distribution_log}
\end{figure}

\subsection{Personalized network}
\textcolor{red}{Question 5: Create a personalized network of the user whose ID is 1. How many nodes and edges does this personalized network have?}

\begin{figure}[h]
\centering
\scalebox{0.7}{\includegraphics{Figures/hj_04}}
\caption{Personalized network of user with ID $1$}
\label{fig:personalized_net}
\end{figure}

The personalized network is shown in Fig. \ref{fig:personalized_net}. The number of nodes is $348$, and the number of edges is $2866$.\\

\textcolor{red}{Question 6: What is the diameter of the personalized network? Please state a trivial upper and lower bound for the diameter of the personalized network.}

The diameter of the personalized netowrk is $2$. A trivial upper bound of the diameter of the personalized network is $2$ and the lower bound of the personalized network is $1$.\\

\textcolor{red}{Question 7: In the context of the personalized network, what is the meaning of the diameter of the personalized network to be equal to the upper bound you derived in question 6. What is the meaning of the diameter
of the personalized network to be equal to the lower bound you derived in question 6?}

The meaning is that: give the core node of the personalized network, when the number of the neighbor nodes is $1$, clearly the diameter of this network is $1$; If the number of the neighbor nodes is greater than $1$, since all neighbor nodes are connected to the core node, thus the diameter of this network is $2$.


\subsection{Core node’s personalized network}

\subsubsection{Community structure of core node’s personalized network}

\subsubsection{Community structure with the core node removed}

\subsubsection{Characteristic of nodes in the personalized network}


\subsection{Friend recommendation in personalized networks}

\subsubsection{Neighborhood based measure}


\subsubsection{Friend recommendation using neighborhood based measures}

\subsubsection{Creating the list of users}

\subsubsection{Average accuracy of friend recommendation algorithm}





\section{Google+ Network}

\subsection{Community structure of personal networks}




\end{document}







